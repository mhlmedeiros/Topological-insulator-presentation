\PassOptionsToPackage{usenames,dvipsnames}{xcolor}
\documentclass[usenames, dvipsnames]{beamer}
\usepackage[utf8]{inputenc}
\usepackage[brazil]{babel}
\usepackage{graphicx}
\usepackage{multimedia}
\usepackage{smartdiagram}
\usepackage{xcolor}
\usepackage{tikz}
\usetikzlibrary{shapes,arrows}
\usepackage{braket}
\usepackage{dsfont}
\usepackage{amsmath,amssymb}
\usepackage{empheq}
\usepackage[most]{tcolorbox}
\usepackage{subfig}
\usepackage{verbatim}
\usepackage{hyperref}
 
\newtcbox{\mymath}[1][]{%
    nobeforeafter, math upper, tcbox raise base,
    enhanced, colframe=blue!50!green!50,
    colback=blue!10!white, boxrule=1pt,
    #1}
\newtcbox{\mymathred}[1][]{%
    nobeforeafter, math upper, tcbox raise base,
    enhanced, colframe=red!50!black!50,
    colback=red!10!white, boxrule=1pt,
    #1}
 
%\usetheme{Bergen}
%\usetheme{Berkeley}
%\usetheme{Copenhagen}
%\usetheme{CambridgeUS}
%\usetheme{Szeged}
%\usetheme{Goettingen}
\usetheme{bjeldbak/beamerthemebjeldbak}
%\usecolortheme{seahorse}
% \usecolortheme{beaver}

% O bloco de comandos a seguir garante que a cada início de seção
% surge o sumário com a seção corrente em destaque  
%
%\AtBeginSection[]
%{
%  \begin{frame}
%    \frametitle{Sumário}
%    \tableofcontents[currentsection]
%  \end{frame}
%}


\title[introduction to topological insulators] %optional
{Topological Insulators}
\subtitle{A short introduction}

\author[My name]{\textbf {Marcos H. Lima de Medeiros}\\ \textbf{\footnotesize Supervisor:} \footnotesize Luis Gregório Dias da Silva }
\institute[VFU] % (optional)
{
  Instituto de Física\\
  Universidade de São Paulo
}
\date[VLC 2013] % (optional)
{October, 2018}
%\logo{\includegraphics[height=0.6cm]{usp.png}}





\begin{document}
\frame{\titlepage}

\begin{frame}
 \frametitle{Summary}
 \tableofcontents[pausesections]
\end{frame}


\begin{frame}
\frametitle{Textos em destaque}

In this slide, some important text will be
\alert{highlighted} beause it's important.
Please, don't abuse it.

\begin{block}{Remark}
Sample text
\end{block}

\begin{alertblock}{Important theorem}
Sample text in red box
\end{alertblock}

\begin{examples}{Exemplo}
Sample text in green box. "Examples" is fixed as block title.
\end{examples}

\end{frame}


\begin{frame}
\frametitle{Two-column slide}

\begin{columns}

\column{0.5\textwidth}
This is a text in first column.
$$ E= mc^2 $$
\begin{itemize}
\item First item
\item Second item
\end{itemize}

\column{0.5\textwidth}
This text will be in the second column
and on a second tought this is a nice looking
layout in some cases.
\end{columns}
\end{frame}


\end{document}


























